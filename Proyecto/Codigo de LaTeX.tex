%%%%%%%%%%%%%%%%%%%%%%%%%%%%%%%%%%%%%%%%%%%%%%%%%%%%%%%%%%%%%%%%%%%%%%%%%%%%
% AGUJournalTemplate.tex: this template file is for articles formatted with LaTeX
%
% This file includes commands and instructions
% given in the order necessary to produce a final output that will
% satisfy AGU requirements, including customized APA reference formatting.
%
% You may copy this file and give it your
% article name, and enter your text.
%
%
% Step 1: Set the \documentclass
%
%

%% To submit your paper:
\documentclass[draft]{agujournal2019}
\usepackage{lineno}
\usepackage{url} %this package should fix any errors with URLs in refs.
\usepackage[inline]{trackchanges} %for better track changes. finalnew option will compile document with changes incorporated.
\usepackage{soul}
\linenumbers
%%%%%%%
% As of 2018 we recommend use of the TrackChanges package to mark revisions.
% The trackchanges package adds five new LaTeX commands:
%
%  \note[editor]{The note}
%  \annote[editor]{Text to annotate}{The note}
%  \add[editor]{Text to add}
%  \remove[editor]{Text to remove}
%  \change[editor]{Text to remove}{Text to add}
%
% complete documentation is here: http://trackchanges.sourceforge.net/
%%%%%%%

\draftfalse

%% Enter journal name below.
%% Choose from this list of Journals:
%
% JGR: Atmospheres
% JGR: Biogeosciences
% JGR: Earth Surface
% JGR: Oceans
% JGR: Planets
% JGR: Solid Earth
% JGR: Space Physics
% Global Biogeochemical Cycles
% Geophysical Research Letters
% Paleoceanography and Paleoclimatology
% Radio Science
% Reviews of Geophysics
% Tectonics
% Space Weather
% Water Resources Research
% Geochemistry, Geophysics, Geosystems
% Journal of Advances in Modeling Earth Systems (JAMES)
% Earth's Future
% Earth and Space Science
% Geohealth
%
% ie, \journalname{Water Resources Research}

\journalname{Universidad Nacional de Colombia - Facultad de ingeniería}


\begin{document}

%% ------------------------------------------------------------------------ %%
%  Title
%
% (A title should be specific, informative, and brief. Use
% abbreviations only if they are defined in the abstract. Titles that
% start with general keywords then specific terms are optimized in
% searches)
%
%% ------------------------------------------------------------------------ %%

% Example: \title{This is a test title}

\title{Estudio de la disponibilidad de recursos hídricos en el municipio de Subachoque}

%% ------------------------------------------------------------------------ %%
%
%  AUTHORS AND AFFILIATIONS
%
%% ------------------------------------------------------------------------ %%

% Authors are individuals who have significantly contributed to the
% research and preparation of the article. Group authors are allowed, if
% each author in the group is separately identified in an appendix.)

% List authors by first name or initial followed by last name and
% separated by commas. Use \affil{} to number affiliations, and
% \thanks{} for author notes.
% Additional author notes should be indicated with \thanks{} (for
% example, for current addresses).

% Example: \authors{A. B. Author\affil{1}\thanks{Current address, Antartica}, B. C. Author\affil{2,3}, and D. E.
% Author\affil{3,4}\thanks{Also funded by Monsanto.}}

\authors{MANUEL NIÑO}


% \affiliation{1}{First Affiliation}
% \affiliation{2}{Second Affiliation}
% \affiliation{3}{Third Affiliation}
% \affiliation{4}{Fourth Affiliation}

\affiliation{1}{Universidad Nacional de Colombia}
%(repeat as many times as is necessary)

%% Corresponding Author:
% Corresponding author mailing address and e-mail address:

% (include name and email addresses of the corresponding author.  More
% than one corresponding author is allowed in this LaTeX file and for
% publication; but only one corresponding author is allowed in our
% editorial system.)

% Example: \correspondingauthor{First and Last Name}{email@address.edu}

\correspondingauthor{Manuel Niño}{mninos@unal.edu.co}

%% Keypoints, final entry on title page.

%  List up to three key points (at least one is required)
%  Key Points summarize the main points and conclusions of the article
%  Each must be 140 characters or fewer with no special characters or punctuation and must be complete sentences

% Example:
% \begin{keypoints}
% \item	List up to three key points (at least one is required)
% \item	Key Points summarize the main points and conclusions of the article
% \item	Each must be 140 characters or fewer with no special characters or punctuation and must be complete sentences
% \end{keypoints}

%% ------------------------------------------------------------------------ %%
%
%  ABSTRACT and PLAIN LANGUAGE SUMMARY
%
% A good Abstract will begin with a short description of the problem
% being addressed, briefly describe the new data or analyses, then
% briefly states the main conclusion(s) and how they are supported and
% uncertainties.

% The Plain Language Summary should be written for a broad audience,
% including journalists and the science-interested public, that will not have 
% a background in your field.
%
% A Plain Language Summary is required in GRL, JGR: Planets, JGR: Biogeosciences,
% JGR: Oceans, G-Cubed, Reviews of Geophysics, and JAMES.
% see http://sharingscience.agu.org/creating-plain-language-summary/)
%
%% ------------------------------------------------------------------------ %%

%% \begin{abstract} starts the second page

\section{abstract}
[Este artículo aborda el problema de la falta de información sobre la disponibilidad de fuentes de agua en el municipio de Subachoque, lo que resulta en una gestión deficiente del recurso hídrico. Como consecuencia, los habitantes de la parte baja de la cuenca del río Subachoque enfrentan escasez de agua, especialmente durante ciertas épocas del año, lo que afecta sus actividades agropecuarias. El estudio tiene como objetivo analizar en detalle esta problemática y proponer soluciones para una gestión más eficaz del agua en la región.]

\section*{Palabras clave}
[Gestión, Modelo estadístico, agua subterránea, agua superficial]


%% ------------------------------------------------------------------------ %%
%
%  TEXT
%
%% ------------------------------------------------------------------------ %%

%%% Suggested section heads:
% \section{Introduction}
%
% The main text should start with an introduction. Except for short
% manuscripts (such as comments and replies), the text should be divided
% into sections, each with its own heading.

% Headings should be sentence fragments and do not begin with a
% lowercase letter or number. Examples of good headings are:

% \section{Materials and Methods}
% Here is text on Materials and Methods.
%
% \subsection{A descriptive heading about methods}
% More about Methods.
%
% \section{Data} (Or section title might be a descriptive heading about data)
%
% \section{Results} (Or section title might be a descriptive heading about the
% results)
%
% \section{Conclusions}


\section{Introducción}
La gestión, análisis y obtención de datos referentes a los recursos hídricos de una región es importante para dar un manejo adecuado a estos y evitar problemáticas como desabastecimiento o afectación a los ecosistemas. Este es el caso del municipio de Subachoque, en donde a pesar de tener disponibilidad de información proveniente de las diferentes estaciones del IDEAM, no existe una caracterización y/o estimación del uso del agua proveniente de diferentes fuentes en las actividades agropecuarias. 
%Text here ===>>>


%%

%  Numbered lines in equations:
%  To add line numbers to lines in equations,
%  \begin{linenomath*}
%  \begin{equation}
%  \end{equation}
%  \end{linenomath*}



%% Enter Figures and Tables near as possible to where they are first mentioned:
%
% DO NOT USE \psfrag or \subfigure commands.
%
% Figure captions go below the figure.
% Table titles go above tables;  other caption information
%  should be placed in last line of the table, using
% \multicolumn2l{$^a$ This is a table note.}
%
%----------------
% EXAMPLE FIGURES
%
% \begin{figure}
% \includegraphics{example.png}
% \caption{caption}
% \end{figure}
%
% Giving latex a width will help it to scale the figure properly. A simple trick is to use \textwidth. Try this if large figures run off the side of the page.
% \begin{figure}
% \noindent\includegraphics[width=\textwidth]{anothersample.png}
%\caption{caption}
%\label{pngfiguresample}
%\end{figure}
%
%
% If you get an error about an unknown bounding box, try specifying the width and height of the figure with the natwidth and natheight options. This is common when trying to add a PDF figure without pdflatex.
% \begin{figure}
% \noindent\includegraphics[natwidth=800px,natheight=600px]{samplefigure.pdf}
%\caption{caption}
%\label{pdffiguresample}
%\end{figure}
%
%
% PDFLatex does not seem to be able to process EPS figures. You may want to try the epstopdf package.
%

%
% ---------------
% EXAMPLE TABLE
% Please do NOT include vertical lines in tables
% if the paper is accepted, Wiley will replace vertical lines with white space
% the CLS file modifies table padding and vertical lines may not display well
%


%% SIDEWAYS FIGURE and TABLE
% AGU prefers the use of {sidewaystable} over {landscapetable} as it causes fewer problems.
%
% \begin{sidewaysfigure}
% \includegraphics[width=20pc]{figsamp}
% \caption{caption here}
% \label{newfig}
% \end{sidewaysfigure}
%
%  \begin{sidewaystable}
%  \caption{Caption here}
% \label{tab:signif_gap_clos}
%  \begin{tabular}{ccc}
% one&two&three\\
% four&five&six
%  \end{tabular}
%  \end{sidewaystable}

%% If using numbered lines, please surround equations with \begin{linenomath*}...\end{linenomath*}
%\begin{linenomath*}
%\begin{equation}
%y|{f} \sim g(m, \sigma),
%\end{equation}
%\end{linenomath*}

%%% End of body of article

%%%%%%%%%%%%%%%%%%%%%%%%%%%%%%%%
%% Optional Appendix goes here
%
% The \appendix command resets counters and redefines section heads
%
% After typing \appendix
%
%\section{Here Is Appendix Title}
% will show
% A: Here Is Appendix Title
%
%\appendix
%\section{Here is a sample appendix}

%%%%%%%%%%%%%%%%%%%%%%%%%%%%%%%%%%%%%%%%%%%%%%%%%%%%%%%%%%%%%%%%
%
% Optional Glossary, Notation or Acronym section goes here:
%
%%%%%%%%%%%%%%
% Glossary is only allowed in Reviews of Geophysics
%  \begin{glossary}
%  \term{Term}
%   Term Definition here
%  \term{Term}
%   Term Definition here
%  \term{Term}
%   Term Definition here
%  \end{glossary}

%
%%%%%%%%%%%%%%
% Acronyms
%   \begin{acronyms}
%   \acro{Acronym}
%   Definition here
%   \acro{EMOS}
%   Ensemble model output statistics
%   \acro{ECMWF}
%   Centre for Medium-Range Weather Forecasts
%   \end{acronyms}

%
%%%%%%%%%%%%%%
% Notation
%   \begin{notation}
%   \notation{$a+b$} Notation Definition here
%   \notation{$e=mc^2$}
%   Equation in German-born physicist Albert Einstein's theory of special
%  relativity that showed that the increased relativistic mass ($m$) of a
%  body comes from the energy of motion of the body—that is, its kinetic
%  energy ($E$)—divided by the speed of light squared ($c^2$).
%   \end{notation}



\section{Planteamiento del problema}
La gestión del agua en el municipio presenta una problemática significativa, principalmente en relación con el manejo del principal afluente, el río Subachoque. Esta situación ha provocado un desabastecimiento de agua para el uso agropecuario, que constituye la principal fuente de recursos en la región. Los efectos de esta escasez recaen especialmente en los habitantes ubicados aguas abajo de la cuenca.

La raíz de este problema se encuentra en la falta de diversificación en las fuentes de agua, especialmente la subutilización de los acuíferos presentes en las formaciones geológicas locales. Esta limitación se debe, en gran medida, a la falta de información disponible sobre la oferta hídrica de cada una de estas fuentes, tanto superficiales como subterráneas. Además, se observa un desaprovechamiento de los datos recopilados por las estaciones del IDEAM y los estudios realizados por la Corporación Autónoma Regional (CAR), que podrían utilizarse para estimar caudales y volúmenes de agua futuros, teniendo en cuenta su variación temporal.

Esta situación ha generado una ausencia de regulación en el uso del agua del río Subachoque por parte de los usuarios ubicados en la parte alta de la cuenca. Como resultado, se han registrado niveles bajos en el afluente, con impactos negativos tanto para los habitantes locales como para la biodiversidad del ecosistema. Además, se ha puesto en riesgo el abastecimiento de agua potable en los municipios vecinos, que dependen del río para este fin.

Un ejemplo concreto que ilustra esta problemática se encuentra en la zona de la vereda La Cuesta, donde la construcción de un vertedero ha exacerbado la situación. Esta estructura impide el paso del agua cuando el caudal disminuye por debajo de cierto nivel, lo que genera conflictos y quejas por parte de la comunidad local. La falta de caracterización del agua subterránea por parte de las autoridades municipales ha impedido su aprovechamiento por parte de los campesinos, quienes en ocasiones recurren al agua del río de manera indiscriminada, sin considerar las repercusiones aguas abajo.

\section{Estado del arte}
La Corporación Autónoma Regional de Cundinamarca ha llevado a cabo estudios para estimar la disponibilidad de aguas subterráneas y la variación de sus niveles en el área de la Sabana de Bogotá y el área critica (una zona de la sabana de Bogotá en donde se tiene alta demanda de agua subterránea debido a las actividades industriales), incluyendo el municipio de Subachoque y la subcuenca del río homónimo que tiene un área de más de 400 km2. En esta cuenca, se tienen instalados 55 piezómetros que registran la explotación de aproximadamente 44.45 l/s de agua de los acuíferos allí ubicados \cite{CAR1} .

En el estudio de aguas subterráneas elaborado por la CAR, se menciona que el acuífero más monitoreado y que abastece esta cuenca es el ubicado en la formación cuaternario, aunque también se monitorean los acuíferos Guaduas y Guadalupe a través de las mediciones piezométricas. A partir de todos estos datos, la CAR elaboró un plan de manejo de aguas subterráneas que incluye la estimación del balance hídrico en la zona critica con el fin redistribuir y asignar estos caudales teniendo en cuenta el uso del suelo y los POT de cada municipio, caracterizando usuarios y su consumo y analizando y procesando datos para estimar la correlación entre los recursos superficiales y subterráneos.

Aunque se haya propuesto un plan de manejo de aguas subterráneas por parte de la CAR, es una medida que no tiene mucho impacto en el municipio de Subachoque debido a que la mayoría de los piezómetros evaluados pertenecen al municipio de Madrid que cuenta con una industria más desarrollada, lo cual incentiva la caracterización y uso de agua subterránea.

El servicio geológico colombiano (Antiguo INGEOMINAS) realizó un estudio de la cuenca del río Subachoque, en el cual se estimó el balance hídrico de la cuenca, encontrando que, debido a la alta evapotranspiración potencial de la zona, solo ocurre recarga de acuíferos en el mes de octubre, cuando la precipitación es lo suficientemente abundante para que el suelo alcance su capacidad de campo permitiendo la infiltración \cite{INGEOMINAS}. Este es un factor que debe tenerse en cuenta a la hora de explotar el agua subterránea, ya que, lo ideal es garantizar la recarga de acuíferos.

\section{Propuesta de solución}
La solución propuesta para abordar la problemática expuesta implica la elaboración de un plan de gestión integral que incluya la caracterización de los puntos de agua utilizados por agricultores y ganaderos. A partir de la información recopilada en las estaciones limnimétricas distribuidas a lo largo del río, se pretende estimar el caudal extraído para actividades agrícolas, así como determinar el origen del agua utilizada, incluyendo fuentes subterráneas. El objetivo de esta caracterización es estimar de manera precisa la demanda hídrica en la región.

Posteriormente, se propone utilizar datos históricos registrados por las estaciones del Instituto de Hidrología, Meteorología y Estudios Ambientales (IDEAM) y estudios realizados por la Corporación Autónoma Regional (CAR) sobre los acuíferos locales. Estos datos se ajustarán a modelos estadísticos para prever la disponibilidad futura de agua tanto proveniente de fuentes superficiales como subterráneas.

La implementación de este plan de gestión tiene como objetivo primordial la asignación equitativa de un volumen de agua determinado para cada usuario por parte de la autoridad local. Esto garantizará la disponibilidad del recurso para todos los campesinos de la región y contribuirá a la conservación del ecosistema. Además, se busca evitar la completa sequía del río en ciertas épocas del año, lo cual es fundamental para mantener el equilibrio ambiental y asegurar la sostenibilidad a largo plazo.

\section{Conclusiones}
Se evidencia falta de aprovechamiento de las diferentes fuentes de agua en el municipio de Subachoque, provocando una sobreexplotación de este recurso en el río principal de la cuenca. El procesamiento y análisis de los datos obtenidos por las estaciones del IDEAM a lo largo de la cuenca podrían ser de gran ayuda, ya que, al realizar estimaciones de la disponibilidad de agua en las diferentes fuentes a lo largo del tiempo, se puede incentivar el uso de agua que no provenga del río y hacer una mejor gestión del recurso especialmente en época de sequía, ayudando a mitigar los efectos negativos que conlleva la el uso inadecuado del agua del rió Subachoque. 

\bibliographystyle{unsrt}%Used BibTeX style is unsrt
\bibliography{Referencias}


%% ------------------------------------------------------------------------ %%
%% References and Citations

%%%%%%%%%%%%%%%%%%%%%%%%%%%%%%%%%%%%%%%%%%%%%%%
%
% \bibliography{<name of your .bib file>} don't specify the file extension
%
% don't specify bibliographystyle

% In the References section, cite the data/software described in the Availability Statement (this includes primary and processed data used for your research). For details on data/software citation as well as examples, see the Data & Software Citation section of the Data & Software for Authors guidance
% https://www.agu.org/Publish-with-AGU/Publish/Author-Resources/Data-and-Software-for-Authors#citation

%%%%%%%%%%%%%%%%%%%%%%%%%%%%%%%%%%%%%%%%%%%%%%%

%\bibliography{enter your bibtex bibliography filename here}



%Reference citation instructions and examples:
%
% Please use ONLY \cite and \citeA for reference citations.
% \cite for parenthetical references
% ...as shown in recent studies (Simpson et al., 2019)
% \citeA for in-text citations
% ...Simpson et al. (2019) have shown...
%
%
%...as shown by \citeA{jskilby}.
%...as shown by \citeA{lewin76}, \citeA{carson86}, \citeA{bartoldy02}, and \citeA{rinaldi03}.
%...has been shown \cite{jskilbye}.
%...has been shown \cite{lewin76,carson86,bartoldy02,rinaldi03}.
%... \cite <i.e.>[]{lewin76,carson86,bartoldy02,rinaldi03}.
%...has been shown by \cite <e.g.,>[and others]{lewin76}.
%
% apacite uses < > for prenotes and [ ] for postnotes
% DO NOT use other cite commands (e.g., \citet, \citep, \citeyear, \citealp, etc.).
% \nocite is okay to use to add references from your Supporting Information
%



\end{document}



More Information and Advice:

%% ------------------------------------------------------------------------ %%
%
%  SECTION HEADS
%
%% ------------------------------------------------------------------------ %%

% Capitalize the first letter of each word (except for
% prepositions, conjunctions, and articles that are
% three or fewer letters).

% AGU follows standard outline style; therefore, there cannot be a section 1 without
% a section 2, or a section 2.3.1 without a section 2.3.2.
% Please make sure your section numbers are balanced.
% ---------------
% Level 1 head
%
% Use the \section{} command to identify level 1 heads;
% type the appropriate head wording between the curly
% brackets, as shown below.
%
%An example:
%\section{Level 1 Head: Introduction}
%
% ---------------
% Level 2 head
%
% Use the \subsection{} command to identify level 2 heads.
%An example:
%\subsection{Level 2 Head}
%
% ---------------
% Level 3 head
%
% Use the \subsubsection{} command to identify level 3 heads
%An example:
%\subsubsection{Level 3 Head}
%
%---------------
% Level 4 head
%
% Use the \subsubsubsection{} command to identify level 3 heads
% An example:
%\subsubsubsection{Level 4 Head} An example.
%
%% ------------------------------------------------------------------------ %%
%
%  IN-TEXT LISTS
%
%% ------------------------------------------------------------------------ %%
%
% Do not use bulleted lists; enumerated lists are okay.
% \begin{enumerate}
% \item
% \item
% \item
% \end{enumerate}
%
%% ------------------------------------------------------------------------ %%
%
%  EQUATIONS
%
%% ------------------------------------------------------------------------ %%

% Single-line equations are centered.
% Equation arrays will appear left-aligned.

Math coded inside display math mode \[ ...\]
 will not be numbered, e.g.,:
 \[ x^2=y^2 + z^2\]

 Math coded inside \begin{equation} and \end{equation} will
 be automatically numbered, e.g.,:
 \begin{equation}
 x^2=y^2 + z^2
 \end{equation}


% To create multiline equations, use the
% \begin{eqnarray} and \end{eqnarray} environment
% as demonstrated below.
\begin{eqnarray}
  x_{1} & = & (x - x_{0}) \cos \Theta \nonumber \\
        && + (y - y_{0}) \sin \Theta  \nonumber \\
  y_{1} & = & -(x - x_{0}) \sin \Theta \nonumber \\
        && + (y - y_{0}) \cos \Theta.
\end{eqnarray}

%If you don't want an equation number, use the star form:
%\begin{eqnarray*}...\end{eqnarray*}

% Break each line at a sign of operation
% (+, -, etc.) if possible, with the sign of operation
% on the new line.

% Indent second and subsequent lines to align with
% the first character following the equal sign on the
% first line.

% Use an \hspace{} command to insert horizontal space
% into your equation if necessary. Place an appropriate
% unit of measure between the curly braces, e.g.
% \hspace{1in}; you may have to experiment to achieve
% the correct amount of space.


%% ------------------------------------------------------------------------ %%
%
%  EQUATION NUMBERING: COUNTER
%
%% ------------------------------------------------------------------------ %%

% You may change equation numbering by resetting
% the equation counter or by explicitly numbering
% an equation.

% To explicitly number an equation, type \eqnum{}
% (with the desired number between the brackets)
% after the \begin{equation} or \begin{eqnarray}
% command.  The \eqnum{} command will affect only
% the equation it appears with; LaTeX will number
% any equations appearing later in the manuscript
% according to the equation counter.
%

% If you have a multiline equation that needs only
% one equation number, use a \nonumber command in
% front of the double backslashes (\\) as shown in
% the multiline equation above.

% If you are using line numbers, remember to surround
% equations with \begin{linenomath*}...\end{linenomath*}

%  To add line numbers to lines in equations:
%  \begin{linenomath*}
%  \begin{equation}
%  \end{equation}
%  \end{linenomath*}



